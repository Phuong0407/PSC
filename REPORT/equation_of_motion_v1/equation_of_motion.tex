\documentclass[a4paper, 12pt]{book}

\usepackage{libertine}

\usepackage{amsmath}

\usepackage[inner = 2cm, outer = 2cm]{geometry}

\usepackage{subfiles}

\begin{document}

\chapter{Theoretical Part}

\section{Equation of Motions}

We will consider the equation of motion, there are only two force that are important
\begin{align}
    m \vec{a} &= m \vec{g} + \vec{F}_{aerodynamics} + \vec{F}_{inertia} \\
    I \ddot{\omega} &= \vec{M}_{gravity} + \vec{M}_{aerodynamics} + \vec{M}_{inertia}
\end{align}
The inertia force relate to this motion is due to the acceleration of the airplane only, therefore
\begin{align}
    \vec{F}_{inertia} = - m \vec{a}_{airplane}
\end{align}

The aerodynamic force could be projected into two components, the lift and the drag; however, in this general configuration, we will project them generally into  three compments along three axes. First, we have the components of aerodynamic force based on lift, drag and lateral directions:
\begin{align}
    \vec{F}_{aerodynamics} = \dfrac{1}{2} \rho_\infty \| \vec{v}_{ice} - \vec{v}_{air} \| (\vec{v}_{ice} - \vec{v}_{air}) C_{aerodynamics},
\end{align}
where $C_{aerodynamics}$ are all aerodynamic coefficients.

For the sake of simplicity, we will employ $C_x$, $C_y$ and $C_z$ the aerodynamic force coefficients along axes $x$, $y$ and $z$ of fixed-body coordinates system could be written:
\begin{align}
    \vec{F}_{A, x} = \dfrac{1}{2} \rho_\infty \left[ (\dot{x} - u)^2 + (\dot{y} - v)^2 + (\dot{z} - w)^2 \right] C_x \\
    \vec{F}_{A, y} = \dfrac{1}{2} \rho_\infty \left[ (\dot{x} - u)^2 + (\dot{y} - v)^2 + (\dot{z} - w)^2 \right] C_y \\
    \vec{F}_{A, z} = \dfrac{1}{2} \rho_\infty \left[ (\dot{x} - u)^2 + (\dot{y} - v)^2 + (\dot{z} - w)^2 \right] C_z
\end{align}
The aerodynamic moment can be written with the same manner:
\begin{align}
    \vec{M}_{A, x} = \dfrac{1}{2} \rho_\infty \left[ (\dot{x} - u)^2 + (\dot{y} - v)^2 + (\dot{z} - w)^2 \right] H C_x \\
    \vec{M}_{A, y} = \dfrac{1}{2} \rho_\infty \left[ (\dot{x} - u)^2 + (\dot{y} - v)^2 + (\dot{z} - w)^2 \right] H C_y \\
    \vec{M}_{A, z} = \dfrac{1}{2} \rho_\infty \left[ (\dot{x} - u)^2 + (\dot{y} - v)^2 + (\dot{z} - w)^2 \right] H C_z
\end{align}
$H$ is a reference length from ice rectangular cuboid; $u$, $v$ and $w$ are the velocity of flow field with respect to fixed-body coordinate system.

An arbitrary velocity vector is transformed from a fixed coordinates system to body-fixed coordinates system by the following matrix multiplication:
\begin{align}
    \begin{bmatrix}
        \dot{x} \\
        \dot{y} \\
        \dot{z}
    \end{bmatrix}
    =
    \begin{bmatrix}
        c_\theta c_\psi & c_\theta s_\psi & -s_\theta \\
        -c_\phi s_\psi + s_\phi s_\theta c_\psi & c_\phi c_\psi + s_\phi s_\theta s_\psi & s_\phi c_\theta \\
        s_\phi s_\psi + c_\phi s_\theta c_\psi & -s_\phi c_\psi + c_\phi s_\theta s_\psi & c_\phi c_\theta
    \end{bmatrix}\cdot
    \begin{bmatrix}
        \dot{X} \\
        \dot{Y} \\
        \dot{Z}
    \end{bmatrix},
\end{align}
where $(\psi, \theta, \phi)$ are Euler's angle: rotating the fixed coordinates system successively angle $\psi$, $\theta$, $\phi$ about the axes $Z$, $y'$ and $x''$. We denote this matrix as $\underline{\underline{R}}_{\psi, \theta, \phi}$, then we could write the transformation of flow field velocity:
\begin{align}
    \begin{bmatrix}
        u \\
        v \\
        w
    \end{bmatrix}
    = \underline{\underline{R}}_{\psi, \theta, \phi}
    \cdot
    \begin{bmatrix}
        U \\
        V \\
        W
    \end{bmatrix}
\end{align}

Then we could write the equation of motion with the following notation for the rotation of ice cuboid $\boldsymbol{\omega} = \omega_x \vec{e}_x + \omega_y \vec{e}_y + \omega_z \vec{e}_z$:
\begin{align}
    F_x & = m(\dot{x} + \omega_y \dot{z} - \omega_z \dot{y}) \\
    F_y & = m(\dot{y} + \omega_z \dot{x} - \omega_x \dot{x}) \\
    F_z & = m(\dot{z} + \omega_x \dot{y} - \omega_y \dot{z}) \\
    M_x & = I_x\dot{\omega}_x + I_z\omega_z \omega_y - I_y \omega_y \omega_z \\
    M_y & = I_y\dot{\omega}_y + I_x\omega_x \omega_z - I_z\omega_z \omega_x \\
    M_z & = I_z\dot{\omega}_z + I_y\omega_y \omega_x - I_x\omega_x \omega_y
\end{align}
where we choose the origin is at the center of mass, then $I_{xy} = I_{yz} = I_{zx} = 0$. This leads to the last three equations of moment.

\begin{align}
    F_i = F_{A, i} + m g_i \quad\text{and}\quad M_i = M_{A, i}
\end{align}
where
\begin{align}
    \begin{bmatrix}
        g_x \\
        g_y \\
        g_z
    \end{bmatrix} =
    \underline{\underline{R}}_{\psi, \theta, \phi}
    \cdot
    \begin{bmatrix}
        0 \\
        0 \\
        g
    \end{bmatrix}
\end{align}

The rest of the problem is to determine aerodynamic coefficients. This is the suject of the following section.

\end{document}